\documentclass[sigconf, timestamp-false, anonymous=true]{acmart}

\usepackage{listings}

%% Rights management information.  This information is sent to you
%% when you complete the rights form.  These commands have SAMPLE
%% values in them; it is your responsibility as an author to replace
%% the commands and values with those provided to you when you
%% complete the rights form.
\setcopyright{none}

\acmConference[ESEC/FSE '20]{ESEC/FSE '20: ACM Joint European Software Engineering Conference and Symposium 
on the Foundations of Software Engineering}{November 8-13, 2020}{Sacramento, CA, USA}
\acmYear{2020}

\newcommand\todo[1]{\textcolor{red}{#1}}

%%
%% Submission ID.
%% Use this when submitting an article to a sponsored event. You'll
%% receive a unique submission ID from the organizers
%% of the event, and this ID should be used as the parameter to this command.
%%\acmSubmissionID{123-A56-BU3}

%%
%% The majority of ACM publications use numbered citations and
%% references.  The command \citestyle{authoryear} switches to the
%% "author year" style.

%%
%% end of the preamble, start of the body of the document source.
\begin{document}

\title{A Study of Multi-Edit Bug Patches}

%add authors & shortauthors if we are lucky :)

\begin{abstract}
  \todo{note that this is probably full of falsehoods, I'm just thinking by
    typing.} Automatic program repair is a promising approach for reducing the
    cost of quality assurance practices and faulty software. To date, most
    techniques proposed for test-driven automatic repair have succeeded
    primarily on bugs that benefit from short, single-edit patches. Techniques
    that succeed on multi-edit bugs often do so by patching them in an
    alternative, single-edit way, or by targeting particular multi-edit bug
    patterns. Empirical studies of real-world similarly tend to focus on the
    patterns exhibited by single-edit bugs, and have not examined repairability
    of multi-edit bugs in detail. We present a comprehensive empirical analysis
    of multi-edit bugs in open source Java programs, focusing on static and
    dynamic properties that define the repair search space for a given bug (and
    thus, in turn, the challenges that apply to automatically addressing them).
    This analysis focuses on the key challenges of the dynamic program repair
    problem: the \emph{mutations and fix code} used to repair multi-edit bugs;
    the \emph{fault locations} and their relationships; and the \emph{objective
      function}, and in particular how and to what degree test cases can be used
    (or not) to identify partial repairs. We identify key takeaways and
    challenges, with implications for future work in expressive, multi-chunk bug
    repair.
\end{abstract}

%%
%% The code below is generated by the tool at http://dl.acm.org/ccs.cfm.
%% Please copy and paste the code instead of the example below.
\begin{CCSXML}
<ccs2012>
<concept>
<concept_id>10011007.10011074.10011099.10011102</concept_id>
<concept_desc>Software and its engineering~Software defect analysis</concept_desc>
<concept_significance>500</concept_significance>
</concept>
<concept>
<concept_id>10011007.10011074.10011784</concept_id>
<concept_desc>Software and its engineering~Search-based software engineering</concept_desc>
<concept_significance>500</concept_significance>
</concept>
</ccs2012>
\end{CCSXML}

\ccsdesc[500]{Software and its engineering~Software defect analysis}
\ccsdesc[500]{Software and its engineering~Search-based software engineering}

\keywords{software bugs, program repair}

\maketitle

\section{Introduction}

Software bugs are hard. That's why we need automated program repair.

Many bugs require multiple chunks or parts of patches. For example, Sobreira et. al reports that over half of the bugs in Defects4J are patched by two or more chunks ~cite{d4j-dissection}. In Zhong and Su's empirical study of bugs in Apache projects, they found that 70\% of buggy source files required two or more repair actions~\cite{zhong2015}. \todo{(this is, however, different than multiple locations.)}

Despite the prevalence of multi-chunk bugs, much of the previous work on automated program repair works best on bugs with single location patches, as to avoid the combinatorial growth of the search space with multiple locations. For example, fault localization techniques often assume that a bug is localized if any one buggy line is identified~\cite{fl-survey-wong}. Program repair techniques are often most effective on bugs that have only a single repair location~\cite{rsrepair, ae, hdrepair}.

Work on multi-chunk bugs are often focused on code clones and syntactically similar code chunks~\cite{saha2019harnessing,jiang2019cmsuggester}. \todo{something about the empirical multi-entity paper?~\cite{wang2018}} \todo{some more citations from Zhen's intro~\cite{fl-multi-faults, patch-correctness, Le2018}}

In order to repair more complex multi-chunk bugs, we need to study their characteristics. In this work, we hope to look at the characteristics of multi-chunk bugs, with an eye to automatically repairing them.

We focus on three specific aspects -- fault localization, repair actions, and evaluation.


\section{Background}
\subsection{Automated Program Repair (APR)}
Automated program repair (APR) techniques aim to find patches for broken programs. 
Given a buggy program and an oracle for program correctness (e.g: test cases), APR 
techniques attempt to produce a series of edits that causes the program to satisfy the 
correctness oracle. 

A popular search-based repair technique is to treat program repair as an optimization 
problem, where the repair technique attempts uses a heuristic for patch quality to guide 
the search for a high quality patch within the space of possible patches. A popular heuristic
measure of patch quality is the number of passing test cases.

\subsection{Fault Localization}

Work on fault localization started in assisting human developers with debugging. When people started doing program repair, folks just plucked those techniques and used them for program repair. 

The most commonly used fault localization technique is Spectrum Based Fault Localization (SBFL). 

\section{Datasets}
We study bugs from Defects4J~\cite{defects4j} and Bears~\cite{bears}---datasets of 
bugs found in real world software projects.

Defects4J contains 438 bugs from six Java software projects. This  
is currently the most popular Java dataset for evaluating program repair tools~\cite{durieux-repair-them-all}.
Many such tools, however, overfit to Defects4J and perform worse on other 
benchmarks~\cite{durieux-repair-them-all}. 

Bears is a set of Java bugs derived from failed Travis-CI builds of GitHub projects.
Bears offers 251 bugs from 72 software projects, offering a greater diversity of 
projects compared to Defects4J.

\todo{can we have a description of chunk rules here (should be moving the stuff in partial repairs subsection), plus maybe a table or something about how many multichunk bugs there are?}

\section{Fitness Experiment}

In search based program repair, one key area of research is to find ways to measure how "close" a candidate patch is to a full repair (a fitness function),
and this is usually done via measuring unit test performance of the candidate patch. 
Given a buggy program and a valid repair, if we apply a part of the valid repair to the buggy program (a partial repair), then semanticly the partial repair is closer to the full repair compared to the original. 
Ideally, we would want the fitness function to guide the search towards a full repair by identifying partial repairs as "closer" to full repair than original.
However, sometimes partial repairs will perform no different in unit tests compared to the original program, and other times they could perform worse. We want to find out how often each of these situations happen.

Moreover, unit test performance can be measured in different granularity levels. The most common ones used in existing APR tools are class level and method level. 
It is commonly believed that the more granular a fitness functions is, the better it is because it collects more information and is less likely to plateau. Thus, we introduce a third granularity level: assertion level, which is more granular than method level granularity.
We would like to compare the performance of the fitness functions at identifying partial repairs at different granularity levels.

The results may provide valuable information to future search-based program repair tool designs.

\subsection{Partial Repairs}

For each bug in Defects4j (excluding Clojure bugs) and Bears (single-module only), we look at the provided correct patch and divide it into edit chunks using the following steps:

1. Each consecutive block of edits (including both insertion and deletion) is labeled as one chunk.

2. Discard all edits that does not actually affect the program (such as comment change, fixing spaces, etc)

3. If a set of matched brackets that are both inserted or deleted in two different chunk, merge those two chunks into one; if a bracket is deleted in one chunk but inserted back in another chunk (matched to the same opposite bracket), merge these two chunks into one

4. We introduce a new set of edits, Base, initially empty. If a variable declaration is deleted, discard the deletion edit; if a variable declaration is inserted, move this edit to Base; if a variable declaration is moved upwards in code, move the variable moving edits to Base; if a variable declaration is moved downwards in code, discard the moving edits. Declaration modifications (i.e. changing parts of the variable declaration, but not the variable name or the location of the declaration), however, does not belong to any of the cases above.

5. Any insertion of imports or additional helper methods are moved to Base; any deletion of imports or helper methods is discarded

After the steps above, we count the number of remaining chunks (not including Base) and call it the chunk number of the bug, label the chunks with positive numbers, and make a power set of the chunks excluding the empty set and the complete set. Then, for each subset of the chunks, we apply all edits in chunks in this subset and Base to the original buggy code, and we define this as a partial repair. For a bug with chunk number n, it should have $2^n-2$ partial repairs. 

In this experiment, each edit chunk will be viewed as a single edit. Thus, only bugs with chunk number between 2 and 6 (inclusive) is selected, as bugs with chunk number more than 6 are rare but has too many partial repairs to evaluate.

Note that the reason of steps 4 and 5 is to make sure that all declarations are present in all partial repairs such that the partial repairs compile (thus can be tested). Since there are almost no cases where a declaration in patch have side effects, it is okay to include redundant variable/method declarations in partial repairs, as they will not affect the program's performance in unit tests. 

\subsection{Unit Testing Granularity}

There are different ways to compare unit test results, and the most common ways are class-level granularity and method-level granularity. At Class-level granularity, we only look at which test classes passed and which failed; at method-level granularity, we look at which test methods passed and which failed.

Here we introduce a third level of granularity: assertion-level granularity. For each test method $M$, let $A(M)$ be the set of all assert statements in $M$. When $M$ is run, if an assertion failed, the failure is recorded and the method is allowed to continue to run (as opposed to normally the test method throws an error and terminates). After running the method, for each assert statement $a\in A(M)$, let $b(a)$ be 1 if $a$ never failed once during the running of $M$, and 0 otherwise. We define the assertion score of $M$ to be $AS(M)=\frac{\Sigma_{a\in A(M)}b(a)}{|A(M)|}$. If $M$ failed to run to completion due to timeouts or exceptions that are not related to assertions, then we define $AS(M)=0$. Thus by definition, $AS(M)=1$ if $M$ passes. If a program passed more assertions in $M$, there should be an increase in $AS(M)$.

\subsection{Test Result Notations}

For each bug, run unit tests on all three levels of granularity on the original buggy code and all partial repairs. Then, the test results of each partial repair is compared to the test results of the original buggy code, and the comparison result is represented using one of the following labels (PR means partial repair, OBC means original buggy code):

\begin{tabular}{| l | l |}
\hline
  Label & Meaning \\
  
  \hline
  c+ & PR passed more test classes than OBC  \\\hline
  c- & PR passed less test classes than OBC  \\\hline
  m+ & PR and OBC passed same number of test classes, \\
  & PR passed more methods than OBC and passed all methods that OBC passed\\\hline
  m- & PR and OBC passed same number of test classes, \\
  & OBC passed more methods than PR and passed all methods that PR passed\\\hline
  m\~ & PR and OBC passed same number of test classes, \\
  & PR passed some methods that OBC didn't, \\
  & and OBC passed some methods that PR didn't \\\hline
  a+ & PR and OBC passed exact same test methods, \\
  & PR has higher assertion scores in some failed methods than OBC \\
  & and has equal assertion scores in all other failed methods \\\hline
  a- & PR and OBC passed exact same test methods, \\
  & PR has lower assertion scores in some failed methods than OBC \\
  & and has equal assertion scores in all other failed methods \\\hline
  a\~ & PR and OBC passed exact same test methods, \\
  & PR has higher assertion scores in some failed methods than OBC \\
  & and has lower assertion scores in some other failed methods \\\hline
  0 & PR and OBC passed exact same test methods \\
  & and has same assertion score for all failed methods \\ \hline
  NC & PR did not compile \\\hline
 
  
\end{tabular}


\subsection{Unminimized Results}

After doing the steps in section 2.1, there are 97 bugs in Defects4j and 64 bugs in Bears that has chunk number between 2 and 6. Since each bug may have different number of partial repairs, we analyzed the results in two different ways: Unweighted (where each partial repair has weight 1), and weighted (where each bug has a total weight of 1, so if a bug has n partial repairs each of them has weight $\frac{1}{n}$). Here is the full result:

\begin{tabular}{| l | l | l | l | l |} \hline
    Dataset & Defects4j & Defects4j & Bears & Bears  \\ \hline
    Weightedness & Unweighted & Weighted & Unweighted & Weighted \\ \hline
    Total & 898 & 97 & 444 & 64 \\ \hline
    c+ & 102 & 19.84 & 129 & 17.67 \\
    c- & 108 & 6.97 & 18 & 4.29 \\
    m+ & 216 & 20.16 & 38 & 2.36 \\
    m- & 65 & 5.18 & 4 & 1 \\
    m~ & 1 & 0.07 & 0 & 0 \\
    a+ & 45 & 6.79 & 6 & 0.2 \\
    a- & 23 & 0.37 & 21 & 1.94 \\
    a~ & 13 & 0.21 & 1 & 0.5 \\
    0 & 248 & 30.77 & 219 & 30.39 \\
    NC & 77 & 6.62 & 8 & 1.67 \\
    \hline
    
    
    \end{tabular}
    
    Thus, from the data above, we can compute the percentage of partial repairs that are identified positively by each granularity level:
    
    \begin{tabular}{| l | l | l | l | l |} \hline
    Dataset & Defects4j & Defects4j & Bears & Bears  \\ \hline
    Weightedness & Unweighted & Weighted & Unweighted & Weighted \\ \hline
    Class Level Granularity & 11.36\% & 20.46\% & 29.05\% & 27.60\%\\
    Method Level Granularity & 35.41\% & 41.25\% & 37.61\% & 31.29\% \\
    Assertion Level Granularity & 40.42\% & 48.25\% & 38.96 \% & 31.60\% \\
    \hline
    
    \end{tabular}

    Here are the percentage of partial repairs that performs no different at each granularity level:
    
    \begin{tabular}{| l | l | l | l | l |} \hline
    Dataset & Defects4j & Defects4j & Bears & Bears  \\ \hline
    Weightedness & Unweighted & Weighted & Unweighted & Weighted \\ \hline
    Class Level Granularity & 68.04\% & 65.53\% & 65.09\% & 56.84\%\\
    Method Level Granularity & 36.64 \% & 39.33\% & 55.63\% & 51.60\% \\
    Assertion Level Granularity & 27.62\% & 31.73\% & 49.32 \% & 47.48\% \\
    \hline
    
    \end{tabular}
    
    Here are the percentage of partial repairs that performed worse compared to original buggy program at each granularity level:
    
    \begin{tabular}{| l | l | l | l | l |} \hline
    Dataset & Defects4j & Defects4j & Bears & Bears  \\ \hline
    Weightedness & Unweighted & Weighted & Unweighted & Weighted \\ \hline
    Class Level Granularity & 12.03\% & 7.18\% & 4.05\% & 6.70\%\\
    Method Level Granularity & 19.27 \% & 12.52\% & 4.95\% & 8.26\% \\
    Assertion Level Granularity & 21.83\% & 12.90\% & 9.68 \% & 11.29\% \\
    \hline
    
    \end{tabular}
    
    

\subsection{Minimized Results}

During the experiment, we found that sometimes not all edit chunks of a bug is necessary to pass all tests, because one or more of its partial repairs passed all tests. This means that some chunks in the provided correct patch is redundant. 

Sicne we're only interested in edits that are necessary for the repair, we processed the data to ignore redundant chunks and all partial repairs that included them. 32 out of 97 defects4j bugs and 34 out of 64 Bears bugs are affected. Some bugs ends up with chunk number 1 after minimization, so they are not included in the Minimized results. The minimized results include 75 remaining bugs in defects4j and 38 remaining bugs of Bears.

\begin{tabular}{| l | l | l | l | l |} \hline
    Dataset & Defects4j & Defects4j & Bears & Bears  \\ \hline
    Weightedness & Unweighted & Weighted & Unweighted & Weighted \\ \hline
    Total & 566 & 75 & 156 & 38 \\ \hline
    c+ & 30 & 7.33 & 19 & 4.02 \\
    c- & 56 & 5.73 & 17 & 4.12 \\
    m+ & 167 & 20.26 & 18 & 3.47 \\
    m- & 44 & 4.86 & 0 & 0 \\
    m~ & 1 & 0.07 & 0 & 0 \\
    a+ & 31 & 6.83 & 6 & 0.2 \\
    a- & 5 & 0.19 & 13 & 1.67 \\
    a~ & 10 & 0.22 & 1 & 0.5 \\
    0 & 181 & 22.73 & 77 & 22.49 \\
    NC & 41 & 5.77 & 5 & 1.83 \\
    \hline
    
    
    \end{tabular}
    
    Thus, from the data above, we can compute the percentage of partial repairs that are identified positively by each granularity level:
    
    \begin{tabular}{| l | l | l | l | l |} \hline
    Dataset & Defects4j & Defects4j & Bears & Bears  \\ \hline
    Weightedness & Unweighted & Weighted & Unweighted & Weighted \\ \hline
    Class Level Granularity & 5.30\% & 9.78\% & 12.18\% & 10.59\%\\
    Method Level Granularity & 34.81\% & 36.80\% & 23.72\% & 19.71\% \\
    Assertion Level Granularity & 40.28\% & 45.91\% & 27.56 \% & 20.24\% \\
    \hline
    
    \end{tabular}

    Here are the percentage of partial repairs that performs no different at each granularity level:
    
    \begin{tabular}{| l | l | l | l | l |} \hline
    Dataset & Defects4j & Defects4j & Bears & Bears  \\ \hline
    Weightedness & Unweighted & Weighted & Unweighted & Weighted \\ \hline
    Class Level Granularity & 77.56\% & 73.56\% & 73.72\% & 73.75\%\\
    Method Level Granularity & 40.11 \% & 39.95\% & 62.18\% & 64.62\% \\
    Assertion Level Granularity & 31.98\% & 30.30\% & 49.36 \% & 59.19\% \\
    \hline
    
    \end{tabular}
    
    Here are the percentage of partial repairs that performed worse compared to original buggy program at each granularity level:
    
    \begin{tabular}{| l | l | l | l | l |} \hline
    Dataset & Defects4j & Defects4j & Bears & Bears  \\ \hline
    Weightedness & Unweighted & Weighted & Unweighted & Weighted \\ \hline
    Class Level Granularity & 9.89\% & 7.64\% & 10.90\% & 10.84\%\\
    Method Level Granularity & 17.67 \% & 14.13\% & 10.90 \% & 10.84\% \\
    Assertion Level Granularity & 18.55 \% & 14.39 \% & 19.23 \% & 14.44\% \\
    \hline
    
    \end{tabular}
    
\subsection{Interpretations}

1. As expected, the more granular the fitness function gets, the more likely a partial repair gets identified positively or negatively.

2. In both datasets, method level granularity performed a lot better than class level granularity. Assertion level granularity performed better than method level granularity in Defects4j (significant increase in positively identifying partial repairs and minimal increase in negatively identifying partial repairs) and not so well in Bears (almost no increase in positively identifying partial repairs while significantly increasing negatively identifying partial repairs).

3. In general, it seems that unit test results are less sensitive to granularity level in Bears compared to Defects4j. Fitness function with class level granularity performs approximately the same on both datasets in minimized results (and in unminimized results it actually did better in Bears), but as we get more granular, the fitness functions performs much better with Defects4j bugs compared to Bears bugs in both minimized and unminimized results.

\subsection{Limitations}

This experiment breaks up full repairs into chunks and treats each chunk as a single edit action. However, in reality most APR techniques edits a line or part of a line at a time. Also, this experiment only checks whether fitness functions can identify partial repairs, and not concerned with how to come up with these partial repairs (depending on specifics of the APR techniques, some may not be in the search space).

Regardless of its limitations, this experiment is provides valid and valuable information to tackling the challenge of automatically repairing bugs that requires multiple edit actions to fully repair and provides insight in future APR research.

\subsection{Relations between fitness experiment result and fault localization experiment result}

(I'm not sure where these findings belong to (or even if they should be in the paper), so I just toss them here for now)

Negative fitness tend to happen in "Same" category of coverage (8 out of 12 bugs has some partial repair that is negative, another 2 have all their partial repairs as 0). In bugs of category "Overlap" or "Disjoint", all of them has some partial repairs that results in more test methods passed, while only 5 out of 12 "Same" bugs has any partial repair that performs better than original. Moreover, "Disjoint" and "Overlap" category has very few partial repairs that has negative fitness compared to original: only 2 out of 17 Disjoint bugs has some negative partial repair, and only 1 out of 10 Overlap bugs have some negative partial repair. These results are kind of intuitive (we would expect multiple edits in the same location to cause more problems for partial repairs than Disjoint ones)

\section{Dependency Analysis}

\subsection{Motivation}
The presence of control or data dependencies in a patch's edits 
indicates coupling between the dependent edits.
Such coupled edits might need to be repaired simultaneously.
\todo{Some motivating examples might be nice.}
We analyze control and data dependencies in bug patches along 
with their relationship to APR tool success.

\subsection{Research Questions}

RQ1: How many multi-edit patches contain control or data dependencies?

RQ2: Are bugs with dependent edits in the human patch more difficult to automatically repair?

\subsection{Methodology}
A patch contains dependent edits if there exists control or data dependencies 
between added, removed, or changed lines in the pre- or post-patch
source code. For practical reasons, we perform intraprocedural analysis, 
although we heuristically consider function arguments as reads 
and invocations of getter and setter methods as reads and writes.

\subsection{Results}

\begin{table}
{\begin{center}
	\begin{tabular}{l | rrrr | r}
		\toprule
		Dataset & Control & Data & Either & Neither & Total \\
		\midrule
		Defects4J & 123 & 78 & 132 & 171 & 303 \\
		& 40\% & 26\% & 44\% & 56\% & 100\% \\
		Bears & 83 & 63 & 95 & 55 & 150 \\
		& 53\% & 42\% & 63\% & 37\% & 100\% \\
		\midrule
		Combined & 206 & 141 & 227 & 226 & 453 \\
		& 45\% & 31\% & 50\% & 50\% & 100\% \\
		\bottomrule
	\end{tabular}
 \end{center}
}
	\caption{Frequencies of multi-line patches with control and data dependent line edits.}
	\label{tab:dependency-frequencies}
\end{table}

\subsubsection{RQ1} Table~\ref{tab:dependency-frequencies} shows the 
frequencies multi-line patches with control or data dependent edits.

\begin{table}
{\begin{center}
	\begin{tabular}{l | rrrr | r}
            	\toprule
            	& Control & Data & Either & Neither & Total \\
            	\midrule
            	Auto-repaired & 34 & 21 & 39 & 94 & 133 \\
            	Not Auto-repaired & 89 & 57 & 93 & 77 & 170 \\
            	\midrule
            	Total & 123 & 78 & 132 & 171 & 303 \\
            	\bottomrule
	\end{tabular}
 \end{center}
}
	\caption{Frequency of Defects4J multi-line patches with respect to the presence of 
	control/data dependent line edits and whether an APR tool successfully 
	repaired the bug in~\cite{durieux-repair-them-all}.}
	\label{tab:dependency-repair-contingency-table}
\end{table}

\begin{table}
{\begin{center}
	\begin{tabular}{l | r}
            	\toprule
		$P(\mbox{Auto-repaired } | \mbox{ Control Dependent})$ & 28\% \\
		$P(\mbox{Auto-repaired } | \neg \mbox{ Control Dependent})$ & 55\% \\
		$P(\mbox{Auto-repaired } | \mbox{ Data Dependent})$ & 27\% \\
		$P(\mbox{Auto-repaired } | \neg \mbox{ Data Dependent})$ & 50\% \\
		$P(\mbox{Auto-repaired } | \mbox{ Control or Data Dependent})$ & 30\% \\
		$P(\mbox{Auto-repaired } | \mbox{ Not Dependent})$ & 45\% \\
		\bottomrule
	\end{tabular}
 \end{center}
}
	\caption{Percentages of auto-repaired bugs in~\cite{durieux-repair-them-all} 
	that (do not) contain control/data dependent line edits.}
	\label{tab:dependency-repair-percents}
\end{table}

\subsubsection{RQ2} 

Table~\ref{tab:dependency-repair-contingency-table} shows
the frequencies of Defects4J multi-line patches with respect to the presence of 
dependent edits and whether an APR tool successfully auto-repaired the bug in~\cite{durieux-repair-them-all}.
Table~\ref{tab:dependency-repair-percents} shows the percentages of 
auto-repaired bugs with respect to the sets of bugs whose patches contain 
dependent edits.
Using $\chi^2$ tests, we find statistically significant relationships between APR  
success and control, data, and the disjunction of either dependencies 
($p < 0.001$ for all).
We find that bugs whose human patches are free of dependencies are
more likely to be successfully fixed by an APR tool.


% !TeX root = fse2020-multi-edit-bugs.tex

\section{Fault localization} \label{secFL}

%% what are the claims, what are we studying, why are we studying it


Spectrum-based fault localization (SBFL) is the most commonly studied dynamic
fault localization technique, and studies have shown that it is more effective
than other techniques such as Mutation-based fault
localization~\cite{mut-analysis} or dynamic program
slicing~\cite{zou2019empirical}. It is a key first step to characterizing the
\emph{fault space} in automatic program repair, narrowing it to a portion of the
program most likely to correspond to the fault.

Fundamentally, the core assumption underlying SBFL is that \emph{failing tests
  execute buggy portions of the code relatively more often than passing tests.}
Thus, if all failing tests execute a particular line of code, then that line of
code is highly suspicious.  SBFL techniques compute a suspiciousness by
measuring how often a line is executed by failing tests as compared to passing
tests. This suspiciousness score can be calculated a few different ways, but is
typically a linear combination of the passing and failing test coverage. One of
the oldest and commonly studied SBFL technique is Tarantula~\cite{tarantula}. In
Tarantula, the suspiciousness score for a line $s$ is calculated by:

$$\mathit{susp(s)} 
=\frac{\mathit{\%F(s)}}{\mathit{\%F(s)} + \mathit{\%P(s)}}$$

where  $\mathit{\%F(s)}$ and $\mathit{\%P(s)}$ are, respectively, the percentage of failing 
tests and passing tests that execute $s$. There are newer, more effective 
SBFL techniques that calculate this score differently, such as Ochiai~\cite{ochiai} and 
DStar~\cite{wong2013dstar}. Both of these were included in an empirical study 
comparing fault localization techniques and were found to localize a similar set of 
faults~\cite{zou2019empirical}.

Assigning a suspiciousness score to each line of code is well-suited to single
location repair. Indeed, the evaluation of most SBFL techniques asks exactly the
question of interest when considering a technique's suitability for
single-location repair: how often does a given technique assign high scores to 
individual buggy lines of code?

Such evaluations, by and large, do not consider the implications of
suspiciousness scoring in a multi-location repair context.  Instead, evaluations
typically consider a technique ``successful'' if it identifies \emph{any} of a
set of changed lines as highly-ranked or likely-suspicious.  While appropriate
for the question being asked in such evaluations, this does not address
suitability for multi-location program repair.  Identifying one of several buggy
locations is generally inadequate in a context where multiple locations must be
modified. In order to investigate how well the SBFL assumption
applies to tests that identify bugs that require multi-location patches, we ask 
the following research question:


\rqorinsight{2}{How well do multiple tests cover the multiple locations
  implicated in bugs that require multi-location patches?}

We focus especially on bugs that are associated with multiple failing tests: a bug
with only a single failing test trivially and equally implicates all the lines
that test executes.  If multiple tests cover the modified locations well, then
SBFL's core assumption holds and multi-location repair can expect to effectively
make use of the off-the-shelf ranking these techniques currently provide
(indeed, this has been tried~\cite{angelix}). If not---that is, if multiple
tests exercise \emph{different} portions of the buggy code---SBFL off-the-shelf
will by definition be less effective in guiding APR to correctly modifying
multiple buggy locations at once.
Fundamentally, we ask whether multiple failing tests cover exactly the same
patch locations, exactly disjoint patch locations, or some combination.

In addition, for purposes of fault localization, we want to know if we can detect 
whether tests will cover disjoint or same fault locations given only the buggy code. 
Thus, we also ask:

\rqorinsight{3}{Do tests that cover multiple faulty locations also cover multiple code 
locations in general?}

\subsection{Methodology}

Between both datasets, there are 191 total bugs that both require multi-location
patches and contain multiple failing tests. However, we were not able to obtain coverage 
data for one of the bugs in Bears, leaving 190 total bugs: 158 in Defects4J, and 32 in
Bears. 
For each of these bugs, we used JaCoCo\footnote{https://www.eclemma.org/jacoco/}
to determine which code locations in both the buggy and patched versions were executed
(at least once) by each failing test.

To answer RQ2, we used the patch locations that were executed by the tests to categorize 
each bug into three \emph{coverage patterns}, as follows:
\begin{itemize}
\item \emph{disjoint} bugs are those for which no line in the patch is covered by all
failing tests.  Intuitively, these are the bugs for which the core SBFL
assumption is violated.
\item \emph{overlap} bugs are those for which some patch lines are covered
by all failing tests, but some are only covered by a subset of the failing
tests. These bugs also violate the core assumption of SBFL, albeit to a lesser
extent.
\item \emph{identical} bugs are those for which all tests cover the exact same
  set of patch lines.
\end{itemize}

In our experiments, we classify bugs using coverage of the patch lines, as opposed to 
coverage of 
patch locations (where a patch location is considered covered if any line at that patch location 
is 
executed). For the purposes of fault localization, coverage of patch locations is more valuable 
than coverage of patch lines, and we analyzed 66 Defects4J bugs\footnote{All the
bugs in Defects4J version 1; we did not have enough time to do the analysis on bugs added in 
Defects4J version 2.}  and all the Bears bugs, a total of 98 bugs, to see whether categorization 
based on patch line 
coverage matched categorization based on patch location coverage.

To answer RQ3, we took all code locations executed by tests in the buggy version and
calculated the percentage of lines that were executed by all failing tests. 
A lower percentage indicates that the failing tests execute different portions of the buggy 
code, whereas a higher percentage indicates that the failing tests all execute a similar
set of code, corresponding to bugs we expect to be \emph{disjoint} and \emph{identical},
respectively.

After calculating the percentages, we split the bugs based on whether we categorized the bug 
as \emph{disjoint}, \emph{identical}, or \emph{overlap} in the previous experiment.
We qualitatively observed the degree to which the percentage corresponded to its coverage 
pattern. \todo{Use a real statistical measure}

\begin{figure}
	\includegraphics[width=\linewidth]{img/coverage-all.png}
	\caption{Distribution of coverage patterns for bugs with multiple failing
      tests that are repaired with multi-location patches in Bears and Defects4J.}
	\label{fig:coverage-all}
\end{figure}


\begin{figure}
	\begin{subfigure}{\linewidth}
		\includegraphics[width=\linewidth]{img/coverage-d4j.png}
		\caption{Distribution of coverage patterns for Defects4J.}
	\end{subfigure}

\vspace{0.5cm}

	\begin{subfigure}{\linewidth}
		\includegraphics[width=\linewidth]{img/coverage-bears.png}
		\caption{Distribution of coverage patterns for Bears.}
	\end{subfigure}
	\caption{Distribution of coverage patterns by dataset,
          indicating significant differences between the multi-location bugs in
          Bears and Defects4J.}
	\label{fig:coverage-datasets}
\end{figure}

\subsection{Results}

\subsubsection{Distribution of Coverage Patterns} \label{sec:cov_patterns}

Figure~\ref{fig:coverage-all} shows results for the overall distribution,
combining the bugs in both datasets. 29\%
of the bugs were \emph{disjoint}.  Thus, for a significant portion of multi-location bugs,
none of the faulty lines were executed by all failing tests.  
In addition, another 39\% were classified as \emph{overlap}: only some of the
buggy locations were executed by all failing tests. In all, over half of 
the bugs that were both multi-location and multi-test contained edit locations that were 
not executed by all failing test cases.

Note, however, that the behavior varies considerably by dataset;
Figure~\ref{fig:coverage-datasets} shows results. In Defects4J, the patterns all have similar 
numbers of bugs, while in Bears, there are fewer \emph{disjoint} bugs and more 
\emph{overlap} 
bugs.
We hypothesize that this may be due to differences in how the two  
datasets were selected and constructed:
The Defects4J dataset specifically enforces a requirement that the patches in the 
dataset be isolated, i.e., not containing any refactorings or new features, to improve the 
usability of the dataset. The authors specifically chose patches that met this requirement, 
and in some cases, manually isolated the bug themselves~\cite{defects4j}. By contrast, the 
bugs in 
Bears are scraped directly from continuous integration systems, and the 
only requirements for inclusion is that the bug must be reproducible and that
the patch must be written by a human. In addition, Bears was designed to be evolvable 
and relatively easily expanded as a dataset, which is at odds with manual inspection and isolation of 
bugs~\cite{bears}.
Given that these two datasets were designed with different values and demonstrate very 
different behavior, these findings highlight the importance of using diverse datasets in 
evaluating program repair techniques.

Out of the 98 bugs we checked, only seven bugs had differing coverage patterns when 
classified based on coverage of patch lines vs. coverage of patch locations. All seven of these 
bugs 
were classified as \emph{overlap} when classified by line coverage, but were classified as 
\emph{identical} when classifying by patch coverage, indicating that all the failing tests were 
executing different paths within the same patch locations. \todo{Is this discussion confusingly 
worded? Because patch locations refer to the patch chunk, made up of one or more lines of 
code.}

Overall, SBFL assumes that faulty locations are executed more often by identifying 
or failing test cases and is not designed to find many of these multi-location
faults. Our results suggest that off-the-shelf SBFL techniques are not
well-suited to guiding APR techniques that conform to the dominant paradigm to repairing
these types of multi-location, multi-test bugs.

\begin{figure}
	\includegraphics[width=.9\linewidth,left]{img/coverage-buggy.png}
	\caption{Boxplots representing the distribution of bugs based on it's percentage of lines 
	executed 
	by all failing tests. A bug scored at 100\% indicates that all failing tests executed the same 
	lines of 
	code, whereas a bug scored at 0\% indicates failing tests all executed different lines of 
	code. 
	These boxplots are split by coverage pattern, as categorized before.}
	\label{fig:coverage-buggy}
\end{figure}

\subsubsection{Identification of Coverage Patterns} \todo{There's a more accurate word 
here 
than "identification"}

Our results are shown in Figure \ref{fig:coverage-buggy}. Here, we see three distributions, 
separated by coverage pattern, and plotted based on the percentage of lines executed by all 
failing tests. We can see some qualitative differences between the three distributions. In 
particular, \emph{identical} bugs are more likely to have failing tests that execute the same lines 
of code, as we might expect. However, in practice, it would be difficult to determine the 
coverage pattern of a bug based on the coverage of failing tests alone, as all the distributions 
range from 5\% to 100\% (note that a bug categorized as disjoint can be scored 100\%, as the 
patch can introduce if statements that break the control flow and cause certain patch locations 
not to be executed). Thus, more work may be needed to identify the different coverage patterns 
from the buggy program.

\subsection{Symptoms}

\rqorinsight{RQ?: Do certain symptoms correlate with multi-edit patches?} 

Since most program repair uses failing tests to identify a fault, correlating 
symptoms with certain classes of repairs may provide a starting point for fault 
localization. We specifically looked at whether certain classes of symptoms correlated with 
multi-edit or single-edit patch.

We define symptoms as the output given in failing test cases. In Java, most of 
these symptoms are exceptions (including JUnit exceptions) and the accompanying error message.

To find a correlation, we recorded the incidence of each symptom among multi-edit and single-edit 
bugs, and then fit the data to a linear regression.

Since the sample sizes are so small, we needed to categorize the symptoms in larger groups in order 
to find statistical significance. We experimented with three groupings:

\todo{what's an alternative to just listing the groupings? and appendix? linking to the source code on 
github?}
\begin{enumerate}
	\item Group all exceptions together except for assertion exceptions and exceptions for which a message indicates some sort of assertion (in our case, we simply looked for the keyword "expected"). We grouped the assertions into a few different types:
	\begin{itemize}
		\item \lstinline{assert_null} is when the assertion is either expecting null, or got null when it wasn't expecting it.
		\item \lstinline{assert_int} is when the failing assertion was expecting a particular int value.
		\item \lstinline{assert_float} is the same as above, but for floats.
		\item \lstinline{assert_obj_arr_date} is when the assertion is expecting an object address, array of objects, or date object/date string. These are grouped together as commonly found but more complex assertions.
		\item \lstinline{error_expected} is when the failing test expected an exception, error, or warning.
		\item \lstinline{timeout} is when a Junit test times out, but also includes errors like stack overflows or out of memory exceptions.
		\item \lstinline{other_assert} for any other assertion that I couldn't easily categorize or parse. The large bulk of bugs in this category are bugs that had no error message at all; it simply failed with an \lstinline{AssertionError} or \lstinline{StackOverflow}.
		\item \lstinline{other}: all non-assertion exceptions.
	\end{itemize}
	\item Grouping symptoms together in an ad hoc way \todo{sell it better.}
	\begin{itemize}
		\item \lstinline{assert_prim}: assertions that compare to a Java primitive, such as int, float, or boolean.
		\item \lstinline{assert_null}: either expected or actual is null
		\item \lstinline{other_assert}: Asserting to anything that's not clearly a primitive or null.
		\item \lstinline{access}: all the bugs pertaining to wrongfully accessing or invoking certain fields or methods, or problems with classpath.
		\item \lstinline{null_pointer}: null pointer exceptions.
		\item \lstinline{timeout}: when a Junit test times out, but also includes errors like stack overflows or out of memory exceptions.
		\item \lstinline{parsing}: Anything related to parsing, serialization, or type conversion.
		\item \lstinline{other}: everything else
	\end{itemize}
	\item This last grouping is an even coarser version of the previous grouping.
	\begin{itemize}
		\item \lstinline{assert_equal}: Any assertion in which the test expected one value but got another
		\item \lstinline{other_assert}: any other assertion
		\item \lstinline{access}: all the bugs pertaining to wrongfully accessing or invoking certain fields or methods, or problems with classpath.
		\item \lstinline{null_pointer}: null pointer exceptions.
		\item \lstinline{parsing}: Anything related to parsing, serialization, or type conversion.
		\item \lstinline{other}: everything else
	\end{itemize}
\end{enumerate}

\todo{How to present statistics?}

assert only
\begin{lstlisting}[basicstyle=\tiny]
Call:
glm(formula = multi ~ assert_obj_arr_date + assert_int + assert_float +
error_expected + timeout + assert_null + other_assert + other,
family = "binomial", data = symptoms)

Deviance Residuals:
Min       1Q   Median       3Q      Max
-1.1630  -0.8391  -0.6331  -0.5009   1.9894

Coefficients:
Estimate Std. Error z value Pr(>|z|)
(Intercept)             -1.46956    0.31792  -4.622 3.79e-06 ***
assert_obj_arr_dateTRUE  0.63225    0.47659   1.327  0.18464
assert_intTRUE          -0.18226    0.38280  -0.476  0.63399
assert_floatTRUE         1.42011    0.44206   3.212  0.00132 **
error_expectedTRUE      -0.36046    0.48704  -0.740  0.45924
timeoutTRUE              0.50710    0.59237   0.856  0.39197
assert_nullTRUE          0.82904    0.37982   2.183  0.02905 *
other_assertTRUE        -0.03597    0.30388  -0.118  0.90577
otherTRUE                0.60662    0.31835   1.906  0.05671 .
---
Signif. codes:  0 '***' 0.001 '**' 0.01 '*' 0.05 '.' 0.1 ' ' 1

(Dispersion parameter for binomial family taken to be 1)

Null deviance: 725.44  on 645  degrees of freedom
Residual deviance: 697.26  on 637  degrees of freedom
AIC: 715.26

Number of Fisher Scoring iterations: 4
\end{lstlisting}

grouping 1
\begin{lstlisting}[basicstyle=\tiny]
Call:
glm(formula = multi ~ access + assert_prim + null_pointer + timeout +
assert_null + parsing + other_assert + other, family = "binomial",
data = symptoms)

Deviance Residuals:
Min       1Q   Median       3Q      Max
-1.3448  -0.7293  -0.6413  -0.6152   1.9233

Coefficients:
Estimate Std. Error z value Pr(>|z|)
(Intercept)      -1.15825    0.30011  -3.859 0.000114 ***
accessTRUE        0.49743    0.35565   1.399 0.161918
assert_primTRUE   0.28861    0.31792   0.908 0.363983
null_pointerTRUE -0.20119    0.45263  -0.444 0.656682
timeoutTRUE       0.23758    0.58290   0.408 0.683582
assert_nullTRUE   0.57468    0.37483   1.533 0.125236
parsingTRUE       0.96885    0.41871   2.314 0.020675 *
other_assertTRUE -0.31892    0.28866  -1.105 0.269231
otherTRUE        -0.09137    0.40435  -0.226 0.821223
---
Signif. codes:  0 '***' 0.001 '**' 0.01 '*' 0.05 '.' 0.1 ' ' 1

(Dispersion parameter for binomial family taken to be 1)

Null deviance: 725.44  on 645  degrees of freedom
Residual deviance: 703.98  on 637  degrees of freedom
AIC: 721.98

Number of Fisher Scoring iterations: 4
\end{lstlisting}

grouping 2
\begin{lstlisting}[basicstyle=\tiny]
Call:
glm(formula = multi ~ assert_equal + access + null_pointer +
parsing + other_assert + other, family = "binomial", data = symptoms)

Deviance Residuals:
Min       1Q   Median       3Q      Max
-1.2102  -0.7915  -0.6497  -0.6220   1.8644

Coefficients:
Estimate Std. Error z value Pr(>|z|)
(Intercept)      -1.24535    0.29793  -4.180 2.92e-05 ***
assert_equalTRUE  0.24537    0.28801   0.852  0.39424
accessTRUE        0.55033    0.35105   1.568  0.11696
null_pointerTRUE -0.09632    0.45470  -0.212  0.83224
parsingTRUE       1.07674    0.41778   2.577  0.00996 **
other_assertTRUE -0.20294    0.27965  -0.726  0.46804
otherTRUE         0.07740    0.37084   0.209  0.83466
---
Signif. codes:  0 '***' 0.001 '**' 0.01 '*' 0.05 '.' 0.1 ' ' 1

(Dispersion parameter for binomial family taken to be 1)

Null deviance: 725.44  on 645  degrees of freedom
Residual deviance: 709.62  on 639  degrees of freedom
AIC: 723.62

Number of Fisher Scoring iterations: 4
\end{lstlisting}

\paragraph{Results}
In assert only, \lstinline{assert_float} correlates with multiedit bugs. In addition, \lstinline{assert_null} (expecting null or getting null) correlates somewhat with multiedit bugs.

In the two groupings, there are slight to medium correlations between parsing errors (errors in which there is some sort of parsing or conversion of text or objects) and multiedit bugs.

\rqorinsight{RQ?: Do certain symptoms correlate with repair success by existing repair tools?} 
\todo{where to fit this, it's not exactly localization.}

Looking at symptoms that are more or less likely to be repaired by existing repair tools may give us 
insight to the strengths and weaknesses of existing repair tools, and provides insight to opportunities 
to improve program repair tools.

Classifying multi-edit patches using the same symptom categorization previously outlined, we 
performed a Fisher's Exact Test for each symptom category to see if a bug exhibiting that class of 
symptom would be more or less likely to be repaired by automated repair tools than a bug that is not 
exhibiting that symptom. We used the RepairThemAll experiment in order to determine whether a 
bug could be repaired by automated repair tools~\cite{durieux-repair-them-all}.

\paragraph{Results}
We found statistically significant results for symptoms categorized as parsing errors, particularly in 
Defects4J. Table \ref{tab:parsing-repair-frequencies-d4j} shows a strong correlation between bugs 
that exhibit a parsing error and bugs that are unable to be repaired by APR tools. This correlation 
still holds when looking at Defects4J and Bears bugs together, as shown in Table 
\ref{tab:parsing-repair-frequencies-both}. However, in Table 
\ref{tab:parsing-repair-frequencies-bears}, the parsing symptom does not correlate with 
repairability. \todo{Is repairability a word I can use here?}

\todo{We found a difference between bears and d4j in coverage experiments as well as symptoms 
vs. repairability. Is the difference between datasets something we want to dwell on? Should I look at 
the difference between bears and d4j for symptoms vs multi-edit?}

\begin{table}
{\begin{center}
	\begin{tabular}{l | rr | r}
            	\toprule
            	& Symptomatic & Asymptomatic & Total \\
            	\midrule
            	Auto-repaired & 0 & 133 & 133 \\
            	Not Auto-repaired & 12 & 158 & 170 \\
            	\midrule
            	Total & 12 & 291 & 303\\
            	\bottomrule
	\end{tabular}
 \end{center}
}
	\caption{Frequency of Defects4J multi-line patches with respect to the presence of
	the parsing symptom and whether an APR tool successfully
	repaired the bug in~\cite{durieux-repair-them-all}.
	A Fisher's Exact Test found a statistically significant relationship
	($p = 0.001$).}
	\label{tab:parsing-repair-frequencies-d4j}
\end{table}

\begin{table}
{\begin{center}
	\begin{tabular}{l | rr | r}
            	\toprule
            	& Symptomatic & Asymptomatic & Total \\
            	\midrule
            	Auto-repaired & 2 & 14 & 16 \\
            	Not Auto-repaired & 10 & 124 & 134 \\
            	\midrule
            	Total & 12 & 138 & 150\\
            	\bottomrule
	\end{tabular}
 \end{center}
}
	\caption{Frequency of Bears multi-line patches with respect to the presence of
	the parsing symptom and whether an APR tool successfully
	repaired the bug in~\cite{durieux-repair-them-all}.
	A Fisher's Exact Test did not find a statistically significant relationship
	($p = 0.619$).}
	\label{tab:parsing-repair-frequencies-bears}
\end{table}

\begin{table}
{\begin{center}
	\begin{tabular}{l | rr | r}
            	\toprule
            	& Symptomatic & Asymptomatic & Total \\
            	\midrule
            	Auto-repaired & 2 & 147 & 149 \\
            	Not Auto-repaired & 22 & 282 & 304 \\
            	\midrule
            	Total & 24 & 429 & 453\\
            	\bottomrule
	\end{tabular}
 \end{center}
}
	\caption{Frequency of both Defects4 and Bears multi-line patches with respect to the presence of
	the parsing symptom and whether an APR tool successfully
	repaired the bug in~\cite{durieux-repair-them-all}.
	A Fisher's Exact Test found a statistically significant relationship
	($p = 0.007$).}
	\label{tab:parsing-repair-frequencies-both}
\end{table}


\section{Related Work}
Zhong and Su~\cite{zhong2015} did this and that.

Wang et al.~\cite{wang2018} did this and that.

Schulte et al.~\cite{schulte} did this and that.

\section{Limitations}

\subsection{Possibly Unnecessary Edits in Patches}

We found instances in our data set where not all edits are required to 
make the faulty program pass all tests. One possibility is that the extra 
edits satisfy untested specifications. Another possibility is that the edits 
do not affect program functionality (e.g.: refactoring edits). We assume 
that all edits are functionally relevant. We intend to study such seemingly
extraneous edits in future work.

\bibliographystyle{ACM-Reference-Format}
\bibliography{references}

\end{document}
\endinput
