\documentclass[sigconf, timestamp-false, anonymous=true]{acmart}

%% Rights management information.  This information is sent to you
%% when you complete the rights form.  These commands have SAMPLE
%% values in them; it is your responsibility as an author to replace
%% the commands and values with those provided to you when you
%% complete the rights form.
\setcopyright{none}

\acmConference[ESEC/FSE '20]{ESEC/FSE '20: ACM Joint European Software Engineering Conference and Symposium 
on the Foundations of Software Engineering}{November 8-13, 2020}{Sacramento, CA, USA}
\acmYear{2020}


%%
%% Submission ID.
%% Use this when submitting an article to a sponsored event. You'll
%% receive a unique submission ID from the organizers
%% of the event, and this ID should be used as the parameter to this command.
%%\acmSubmissionID{123-A56-BU3}

%%
%% The majority of ACM publications use numbered citations and
%% references.  The command \citestyle{authoryear} switches to the
%% "author year" style.

%%
%% end of the preamble, start of the body of the document source.
\begin{document}

\title{A Study of Multi-Edit Bug Patches}

%add authors & shortauthors if we are lucky :)

\begin{abstract}
	Multi-edit bugs are cool.
\end{abstract}

%%
%% The code below is generated by the tool at http://dl.acm.org/ccs.cfm.
%% Please copy and paste the code instead of the example below.
\begin{CCSXML}
<ccs2012>
<concept>
<concept_id>10011007.10011074.10011099.10011102</concept_id>
<concept_desc>Software and its engineering~Software defect analysis</concept_desc>
<concept_significance>500</concept_significance>
</concept>
<concept>
<concept_id>10011007.10011074.10011784</concept_id>
<concept_desc>Software and its engineering~Search-based software engineering</concept_desc>
<concept_significance>500</concept_significance>
</concept>
</ccs2012>
\end{CCSXML}

\ccsdesc[500]{Software and its engineering~Software defect analysis}
\ccsdesc[500]{Software and its engineering~Search-based software engineering}

\keywords{software bugs, program repair}

\maketitle

\section{Introduction}

\section{Background}
\subsection{Automated Program Repair (APR)}
Automated program repair (APR) techniques aim to find patches for broken programs. 
Given a buggy program and an oracle for program correctness (e.g: test cases), APR 
techniques attempt to produce a series of edits that causes the program to satisfy the 
correctness oracle. 

A popular search-based repair technique is to treat program repair as an optimization 
problem, where the repair technique attempts uses a heuristic for patch quality to guide 
the search for a high quality patch within the space of possible patches. A popular heuristic
measure of patch quality is the number of passing test cases.

\subsection{Fault Localization}


\section{Related Work}
Zhong and Su~\cite{zhong2015} did this and that.

Wang et al.~\cite{wang2018} did this and that.

Schulte et al.~\cite{schulte} did this and that.


\bibliographystyle{ACM-Reference-Format}
\bibliography{references}

\end{document}
\endinput
