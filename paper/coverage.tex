\section{Fault localization}

%% what are the claims, what are we studying, why are we studying it

\subsection{Coverage}

\rqorinsight{RQ?: Do failing tests cover different portions of the fault?}

The assumption underlying SBFL is that failing tests cover buggy portions of 
the code more than passing tests. Thus, if all failing tests cover a particular 
line of code, then that line of code is more suspicious. However, if the 
failing tests cover different portions of the buggy code, then this assumption 
no longer holds.

Therefore, for bugs that are signaled through multiple failing tests and 
require a multi-edit patch, do the different failing tests cover exactly the 
same patch locations, exactly disjoint path locations, or some combination?

Between both datasets, there are 60 total bugs that are both multi-edit and 
multi-test -- 45 in Defects4J and 15 in Bears. For each of these, we used 
Jacoco to obtain coverage for each failing test. Then we extracted the extent 
to which each test covers the patch, and we categorized the bug based on one of 
three coverage patterns: \textit{disjoint} for bugs in which no part of the 
patch is covered by all failing tests, \textit{same} for bugs in which all 
tests cover the exact same parts of the patch, and \textit{overlap} for all 
other bugs.

\todo{how do I handle line coverage vs. chunk coverage? 5 bugs is a pretty 
significant amount with these sample sizes.}

\paragraph{Results}
The overall distribution is seen in Figure \ref{fig:coverage-all}. There are 
quite 
a few more disjoint bugs \todo{how to format the words disjoint, same, and 
overlap?} than same or overlap bugs, indicating that a significant portion of 
bugs cannot be localized by only looking at parts of the program that are 
covered by all failing tests.

\begin{figure}
	\includegraphics[width=\linewidth]{img/coverage-all.png}
	\caption{Distribution of coverage patterns for all multi-test and 
	multi-edit bugs in Bears and Defects4J}
	\label{fig:coverage-all}
\end{figure}


If we divide distribution by dataset, we can see even more interesting 
behavior. Figure \ref{fig:coverage-datasets} shows the Defects4J dataset has a 
much higher proportion of "disjoint" bugs than "overlap" bugs, while the Bears 
dataset has many more "overlap" bugs and comparatively few "disjoint" ones. We 
hypothesize that this may be due to differences in how the two datasets were 
minimized, or due to the projects used in the datasets, themselves. 
\todo{Do we have other hypotheses for why they might be different? Look at 
project level, repair types, etc.}

\begin{figure}[h]
	\label{fig:coverage-datasets}
	\begin{subfigure}{\linewidth}
		\includegraphics[width=\linewidth]{img/coverage-d4j.png}
	\end{subfigure}
	\begin{subfigure}{\linewidth}
		\includegraphics[width=\linewidth]{img/coverage-bears.png}
	\end{subfigure}
	\caption{Distribution of coverage patterns divided by dataset. Above is 
	Defects4J, below is Bears.}
\end{figure}

\subsection{Symptoms}

\rqorinsight{RQ?: Do certain symptoms correlate with multi-edit patches?} 

Since most program repair uses failing tests to identify a fault, correlating 
symptoms with certain classes of repairs may provide a starting point for fault 
localization. We specifically looked at whether certain classes of symptoms correlated with 
multi-edit or single-edit patch.

We define symptoms as the output given in failing test cases. In Java, most of 
these symptoms are exceptions (including JUnit exceptions) and the accompanying error message.

To find a correlation, we recorded the incidence of each symptom among multi-edit and single-edit 
bugs, and then fit the data to a linear regression.

Since the sample sizes are so small, we needed to categorize the symptoms in larger groups in order 
to find statistical significance. We experimented with three groupings:

\todo{what's an alternative to just listing the groupings? and appendix? linking to the source code on 
github?}
\begin{enumerate}
	\item Group all exceptions together except for assertion exceptions and exceptions for which a message indicates some sort of assertion (in our case, we simply looked for the keyword "expected"). We grouped the assertions into a few different types:
	\begin{itemize}
		\item \lstinline{assert_null} is when the assertion is either expecting null, or got null when it wasn't expecting it.
		\item \lstinline{assert_int} is when the failing assertion was expecting a particular int value.
		\item \lstinline{assert_float} is the same as above, but for floats.
		\item \lstinline{assert_obj_arr_date} is when the assertion is expecting an object address, array of objects, or date object/date string. These are grouped together as commonly found but more complex assertions.
		\item \lstinline{error_expected} is when the failing test expected an exception, error, or warning.
		\item \lstinline{timeout} is when a Junit test times out, but also includes errors like stack overflows or out of memory exceptions.
		\item \lstinline{other_assert} for any other assertion that I couldn't easily categorize or parse. The large bulk of bugs in this category are bugs that had no error message at all; it simply failed with an \lstinline{AssertionError} or \lstinline{StackOverflow}.
		\item \lstinline{other}: all non-assertion exceptions.
	\end{itemize}
	\item Grouping symptoms together in an ad hoc way \todo{sell it better.}
	\begin{itemize}
		\item \lstinline{assert_prim}: assertions that compare to a Java primitive, such as int, float, or boolean.
		\item \lstinline{assert_null}: either expected or actual is null
		\item \lstinline{other_assert}: Asserting to anything that's not clearly a primitive or null.
		\item \lstinline{access}: all the bugs pertaining to wrongfully accessing or invoking certain fields or methods, or problems with classpath.
		\item \lstinline{null_pointer}: null pointer exceptions.
		\item \lstinline{timeout}: when a Junit test times out, but also includes errors like stack overflows or out of memory exceptions.
		\item \lstinline{parsing}: Anything related to parsing, serialization, or type conversion.
		\item \lstinline{other}: everything else
	\end{itemize}
	\item This last grouping is an even coarser version of the previous grouping.
	\begin{itemize}
		\item \lstinline{assert_equal}: Any assertion in which the test expected one value but got another
		\item \lstinline{other_assert}: any other assertion
		\item \lstinline{access}: all the bugs pertaining to wrongfully accessing or invoking certain fields or methods, or problems with classpath.
		\item \lstinline{null_pointer}: null pointer exceptions.
		\item \lstinline{parsing}: Anything related to parsing, serialization, or type conversion.
		\item \lstinline{other}: everything else
	\end{itemize}
\end{enumerate}

\todo{How to present statistics?}

assert only
\begin{lstlisting}[basicstyle=\tiny]
Call:
glm(formula = multi ~ assert_obj_arr_date + assert_int + assert_float +
error_expected + timeout + assert_null + other_assert + other,
family = "binomial", data = symptoms)

Deviance Residuals:
Min       1Q   Median       3Q      Max
-1.1630  -0.8391  -0.6331  -0.5009   1.9894

Coefficients:
Estimate Std. Error z value Pr(>|z|)
(Intercept)             -1.46956    0.31792  -4.622 3.79e-06 ***
assert_obj_arr_dateTRUE  0.63225    0.47659   1.327  0.18464
assert_intTRUE          -0.18226    0.38280  -0.476  0.63399
assert_floatTRUE         1.42011    0.44206   3.212  0.00132 **
error_expectedTRUE      -0.36046    0.48704  -0.740  0.45924
timeoutTRUE              0.50710    0.59237   0.856  0.39197
assert_nullTRUE          0.82904    0.37982   2.183  0.02905 *
other_assertTRUE        -0.03597    0.30388  -0.118  0.90577
otherTRUE                0.60662    0.31835   1.906  0.05671 .
---
Signif. codes:  0 '***' 0.001 '**' 0.01 '*' 0.05 '.' 0.1 ' ' 1

(Dispersion parameter for binomial family taken to be 1)

Null deviance: 725.44  on 645  degrees of freedom
Residual deviance: 697.26  on 637  degrees of freedom
AIC: 715.26

Number of Fisher Scoring iterations: 4
\end{lstlisting}

grouping 1
\begin{lstlisting}[basicstyle=\tiny]
Call:
glm(formula = multi ~ access + assert_prim + null_pointer + timeout +
assert_null + parsing + other_assert + other, family = "binomial",
data = symptoms)

Deviance Residuals:
Min       1Q   Median       3Q      Max
-1.3448  -0.7293  -0.6413  -0.6152   1.9233

Coefficients:
Estimate Std. Error z value Pr(>|z|)
(Intercept)      -1.15825    0.30011  -3.859 0.000114 ***
accessTRUE        0.49743    0.35565   1.399 0.161918
assert_primTRUE   0.28861    0.31792   0.908 0.363983
null_pointerTRUE -0.20119    0.45263  -0.444 0.656682
timeoutTRUE       0.23758    0.58290   0.408 0.683582
assert_nullTRUE   0.57468    0.37483   1.533 0.125236
parsingTRUE       0.96885    0.41871   2.314 0.020675 *
other_assertTRUE -0.31892    0.28866  -1.105 0.269231
otherTRUE        -0.09137    0.40435  -0.226 0.821223
---
Signif. codes:  0 '***' 0.001 '**' 0.01 '*' 0.05 '.' 0.1 ' ' 1

(Dispersion parameter for binomial family taken to be 1)

Null deviance: 725.44  on 645  degrees of freedom
Residual deviance: 703.98  on 637  degrees of freedom
AIC: 721.98

Number of Fisher Scoring iterations: 4
\end{lstlisting}

grouping 2
\begin{lstlisting}[basicstyle=\tiny]
Call:
glm(formula = multi ~ assert_equal + access + null_pointer +
parsing + other_assert + other, family = "binomial", data = symptoms)

Deviance Residuals:
Min       1Q   Median       3Q      Max
-1.2102  -0.7915  -0.6497  -0.6220   1.8644

Coefficients:
Estimate Std. Error z value Pr(>|z|)
(Intercept)      -1.24535    0.29793  -4.180 2.92e-05 ***
assert_equalTRUE  0.24537    0.28801   0.852  0.39424
accessTRUE        0.55033    0.35105   1.568  0.11696
null_pointerTRUE -0.09632    0.45470  -0.212  0.83224
parsingTRUE       1.07674    0.41778   2.577  0.00996 **
other_assertTRUE -0.20294    0.27965  -0.726  0.46804
otherTRUE         0.07740    0.37084   0.209  0.83466
---
Signif. codes:  0 '***' 0.001 '**' 0.01 '*' 0.05 '.' 0.1 ' ' 1

(Dispersion parameter for binomial family taken to be 1)

Null deviance: 725.44  on 645  degrees of freedom
Residual deviance: 709.62  on 639  degrees of freedom
AIC: 723.62

Number of Fisher Scoring iterations: 4
\end{lstlisting}

\paragraph{Results}
In assert only, \lstinline{assert_float} correlates with multiedit bugs. In addition, \lstinline{assert_null} (expecting null or getting null) correlates somewhat with multiedit bugs.

In the two groupings, there are slight to medium correlations between parsing errors (errors in which there is some sort of parsing or conversion of text or objects) and multiedit bugs.

\rqorinsight{RQ?: Do certain symptoms correlate with repair success by existing repair tools?} 
\todo{where to fit this, it's not exactly localization.}

Looking at symptoms that are more or less likely to be repaired by existing repair tools may give us 
insight to the strengths and weaknesses of existing repair tools, and provides insight to opportunities 
to improve program repair tools.

Classifying multi-edit patches using the same symptom categorization previously outlined, we 
performed a Fisher's Exact Test for each symptom category to see if a bug exhibiting that class of 
symptom would be more or less likely to be repaired by automated repair tools than a bug that is not 
exhibiting that symptom. We used the RepairThemAll experiment in order to determine whether a 
bug could be repaired by automated repair tools~\cite{durieux-repair-them-all}.

\paragraph{Results}
We found statistically significant results for symptoms categorized as parsing errors, particularly in 
Defects4J. Table \ref{tab:parsing-repair-frequencies-d4j} shows a strong correlation between bugs 
that exhibit a parsing error and bugs that are unable to be repaired by APR tools. This correlation 
still holds when looking at Defects4J and Bears bugs together, as shown in Table 
\ref{tab:parsing-repair-frequencies-both}. However, in Table 
\ref{tab:parsing-repair-frequencies-bears}, the parsing symptom does not correlate with 
repairability. \todo{Is repairability a word I can use here?}

\todo{We found a difference between bears and d4j in coverage experiments as well as symptoms 
vs. repairability. Is the difference between datasets something we want to dwell on? Should I look at 
the difference between bears and d4j for symptoms vs multi-edit?}

\begin{table}
{\begin{center}
	\begin{tabular}{l | rr | r}
            	\toprule
            	& Symptomatic & Asymptomatic & Total \\
            	\midrule
            	Auto-repaired & 0 & 133 & 133 \\
            	Not Auto-repaired & 12 & 158 & 170 \\
            	\midrule
            	Total & 12 & 291 & 303\\
            	\bottomrule
	\end{tabular}
 \end{center}
}
	\caption{Frequency of Defects4J multi-line patches with respect to the presence of
	the parsing symptom and whether an APR tool successfully
	repaired the bug in~\cite{durieux-repair-them-all}.
	A Fisher's Exact Test found a statistically significant relationship
	($p = 0.001$).}
	\label{tab:parsing-repair-frequencies-d4j}
\end{table}

\begin{table}
{\begin{center}
	\begin{tabular}{l | rr | r}
            	\toprule
            	& Symptomatic & Asymptomatic & Total \\
            	\midrule
            	Auto-repaired & 2 & 14 & 16 \\
            	Not Auto-repaired & 10 & 124 & 134 \\
            	\midrule
            	Total & 12 & 138 & 150\\
            	\bottomrule
	\end{tabular}
 \end{center}
}
	\caption{Frequency of Bears multi-line patches with respect to the presence of
	the parsing symptom and whether an APR tool successfully
	repaired the bug in~\cite{durieux-repair-them-all}.
	A Fisher's Exact Test did not find a statistically significant relationship
	($p = 0.619$).}
	\label{tab:parsing-repair-frequencies-bears}
\end{table}

\begin{table}
{\begin{center}
	\begin{tabular}{l | rr | r}
            	\toprule
            	& Symptomatic & Asymptomatic & Total \\
            	\midrule
            	Auto-repaired & 2 & 147 & 149 \\
            	Not Auto-repaired & 22 & 282 & 304 \\
            	\midrule
            	Total & 24 & 429 & 453\\
            	\bottomrule
	\end{tabular}
 \end{center}
}
	\caption{Frequency of both Defects4 and Bears multi-line patches with respect to the presence of
	the parsing symptom and whether an APR tool successfully
	repaired the bug in~\cite{durieux-repair-them-all}.
	A Fisher's Exact Test found a statistically significant relationship
	($p = 0.007$).}
	\label{tab:parsing-repair-frequencies-both}
\end{table}


\subsection{Discussion}
\todo{trying to elicit some takeaways here, not sure if this will stay in.}

We learned:
\begin{itemize}
	\item There are lots of multi-edit bugs in which no portion of the patch is covered by all failing 
	tests.
	\item However, there are also lots of multi-edit bugs that are covered exactly the same way by all 
	failing tests.
	\item Maybe next work includes what makes a disjoint bug disjoint; what makes a same patch 
	same? Correlate with symptoms? oooh.
	\item Speaking of symptoms, symptoms in the parsing category seems to correlate with 
	multi-chunk bugs as well as decreased repair success by existing APR tools, suggesting that ... 
	APR tools don't work well on multi-edit bugs?
\end{itemize}